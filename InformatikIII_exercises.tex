\documentclass[paper=a4, fontsize=11pt]{scrartcl} 
\usepackage[utf8]{inputenc}
\usepackage{amsmath}
\usepackage{amsfonts}
\usepackage{amssymb}
\author{Kim Thuong Ngo}


\usepackage[T1]{fontenc} 
\usepackage{fourier} 

\usepackage{lipsum} 

\usepackage{listings}
\usepackage{graphicx}
\usepackage{tabularx}

\usepackage{sectsty}
\allsectionsfont{\centering \normalfont\scshape} 

\usepackage{fancyhdr} 
\pagestyle{fancyplain} 
\fancyhead{}
\fancyfoot[L]{} 
\fancyfoot[C]{} 
\fancyfoot[R]{\thepage} 
\renewcommand{\headrulewidth}{0pt} 
\renewcommand{\footrulewidth}{0pt}
\setlength{\headheight}{13.6pt}

\numberwithin{equation}{section} 
\numberwithin{figure}{section} 
\numberwithin{table}{section}

\setlength\parindent{0pt} 

\newcommand{\horrule}[1]{\rule{\linewidth}{#1}} 

\title{	
\normalfont \normalsize 
\textsc{Theoretische Informatik} \\ [25pt] 
\horrule{0.5pt} \\[0.4cm] 
\huge Aufgaben \\ 
\horrule{2pt} \\[0.5cm] 
}

\author{Kim Thuong Ngo} 

\date{\normalsize\today} 

%----------------------------------------------------------------------------------------

\begin{document}

\maketitle 

\newpage

\tableofcontents

%----------------------------------------------------------------------------------------

\newpage

\section{Übung 01}

%---------------------------------------------------

\subsection{Definition Sprache}

Geben Sie an, wie Sie einem Mathematiker (der keine Ahnung von theoretischer Informatik hat) erklären würden, was eine Sprache ist. Geben Sie eine formale Definition an. \\

%---------------------------------------------------

\subsection{Alphabet}

Sei $A= \{ a,b,c \}$ ein Alphabet.\\

a) Geben Sie $A^{2}$ an. \\

b) Geben Sie $A^{0}$ an. \\

c) Zeigen Sie, dass $A^{n} \neq A^{n+1}$ für alle $n \epsilon \mathbb{N}$ gilt. \\

%---------------------------------------------------

\subsection{Sprache}

Sei $\sum = \{ a,b,c \}$  und L die Sprache, die aus genau den Wörtern bestehen, die alle 3 Buchstaben des Alphabets mindestens einmal beinhalten. \\

a) Geben SIe 5 Wörter an, die in L enthalten sind. Geben Sie 5 Wörter an, die in $\overline{L} := \sum * \backslash L$ enthalten sind. \\

b) Beweisen Sie: L ist unendlich. \\

c) Beweisen Sie: $\overline{L}$ ist unendlich. \\

%---------------------------------------------------

\subsection{Kombinationen}

Im folgenden sei das Alphabet $\sum = \{ a,b \}$. Geben Sie für alle Kombinationen $L_{i}$, $L_{j}$ ein Wort aus $L_{i} \cap L_{j}$ an, falls $L_{i} \cap L_{j} \neq \emptyset$. Geben Sie außerdem für jeden Schritt an, ob er endlich oder unendlich ist. \\

\begin{itemize}
\item $L_{1}$ = \{ Alle Wörter mit mindestens 2 a's. \} 
\item $L_{2}$ = \{ Alle Wörter mit mindestens einem b. \}
\item $L_{3}$ = \{ Alle Wörter mit einer ungeraden Anzahl an a's. \}
\item $L_{4}$ = \{ Alle Wörter mit Länge höchstens 1 \}
\end{itemize} 
\\

%---------------------------------------------------

\subsection{Wörter}

Sei $\sum = \{ 0,1 \}$ \\

a) Geben Sie an wie viele Wörter der Länge 6 es gibt. \\

b) Geben Sie an, wie viele Sprachen es gibt, die nur Wörter der Länge 6 enthalten. \\

c) Gebe Sie an, wie viele Typ-0-Grammatiken es gibt, die Sprachen erzeugen, welche nur Wörter der Länge 6 enthalten. \\

%----------------------------------------------------------------------------------------

\newpage

\section{PÜ 01}

%---------------------------------------------------

\subsection{Typ-3-Grammatiken}

Geben Sie Typ-3-Grammatiken für folgende Sprachen an ($\sum = \{ a,b \}$): \\

$L_{odd}$ = $\{ w \epsilon \sum *$ | $|w|$ ungerade \} \\

$L_{even}$ = $\{ w \epsilon \sum *$ | $|w|$ gerade  \} \\

$L_{ \leq 3}$ = $\{ w \epsilon \sum *$ | $|w| \leq 3 \} $ \\

$L_{ \geq 3}$ = $\{ w \epsilon \sum *$ | $|w| \geq 3 \} $ \\

$L_{ #(a)odd }$ = $\{ w \epsilon \sum *$ |  Die Anzahl der a's in w ist ungerade \} \\

$L_{ #(a)even }$ = $\{ w \epsilon \sum *$ |  Die Anzahl der a's in w ist gerade \} \\

$L_{1}$ = $\{ w \epsilon \sum *$ | $ #_{a}(w) \equiv 2 (mod 3)  \} $ \\

$L_{2}$ = $\{ w \epsilon \sum *$ | $ #_{a}(w) \equiv #_{b}(w) (mod 3) \} $ \\

$L_{3}$ = $\{ w \epsilon \sum *$ | $ #_{a}(w) < #_{b}(w) \leq 3  \} $ \\

%---------------------------------------------------

\subsection{Typ-2-Grammatiken}

Geben Sie Typ-2-Grammatiken für folgende Sprachen an: \\

1.) $ \sum = \{ a,b \}$, L = \{ $a^{n}b^{n}$ | $n \geq 0$ \} \\

2.) $ \sum = \{ [, ] \}$ , L = \{ w  | w ist ein korrekt geklammerter Ausdruck \} \\

3.) $ \sum = \{ [, ] , (, ) \}$, L = \{ w | w ist ein korrekt geklammerter Ausdruck \} \\

4.) $ \sum = \{ a,b \}$, L = \{ $w \epsilon \sum *$  | $#_{a}(w) = #_{b}(w)$ \} \\

5.) $ \sum = \{ a,b \}$, L = \{ $w \epsilon \sum *$ | $#_{a}(w) = 2 #_{b}(w)$ \} \\

6.) $ \sum = \{ a,b \}$, L = \{ $w \epsilon \sum *$ | $#_{a}(w) > #_{b}(w)$ \} \\

7.) $ \sum = \{ a,b \}$, L = \{ $ww^{R}$ | $w \epsilon \sum *$ \} \\

%---------------------------------------------------------------------------------------

\newpage

\section{Übung 02}

%----------------------------------------------------------------------------------------

\newpage

\section{PÜ 02}

%----------------------------------------------------------------------------------------

\newpage

\section{Übung 03}

%----------------------------------------------------------------------------------------

\newpage

\section{PÜ 03}

%----------------------------------------------------------------------------------------

\newpage

\section{Übung 04}

%----------------------------------------------------------------------------------------

\newpage

\section{PÜ 04}

%---------------------------------------------------------------------------------------
\end{document}